\documentclass[a5paper, 10pt, twoside]{book}

\usepackage[utf8]{inputenc}
\usepackage[T1]{fontenc}

\usepackage[francais]{babel}
%\DecimalMathComma

\usepackage[top=2cm, bottom=2cm, left=2cm, right=2cm]{geometry}

\usepackage{graphicx, wrapfig}

% Chevrons Gauche et Droit
\newcommand{\cg}{\guillemotleft~}
\newcommand{\cd}{~\guillemotright}


%%%%%
\title{Les pégus de la ferme\\Tome I\\Panique à la ferme}
\author{Alexis \bsc{Cabodi}\\\\\cg Comment ça on remarque pas du tout\\mes sources d'inspiration ?!\cd}
\date{}
%%%%%
\begin{document}
\maketitle

\renewcommand{\contentsname}{Sommaire}
\tableofcontents

\vfill % Après sur la prochaine page
Des lectures audio seront prochainement disponibles.


\newpage
\newpage

\chapter*{Carte de Péguland}

\newpage
\newpage


\chapter{La disparition (avec des e)}
\cg Ben la poule !\cd

\begin{wrapfigure}{o}{3cm} % Par le plus grand des hasards, elle est à peu près centrée à côté du paragraphe. Sinon spécifier [12]
\begin{center}
Kidnapoule
\includegraphics[width=3cm]{imgs/trobelpoule.png}
\end{center}
\end{wrapfigure}
Tels étaient les derniers mots que Gertrude entendit de Gérard, avant qu'il se fasse enlever par des oies sauvages nourries et vivant à la ferme en échange de bon lait de poule toutes les semaines. Gertrude reconnu bien là Bernard et Wilfried-Constantin, ses deux poules préférées, qui à elles seules arrivaient à soulever Gérard pour l'emmener on ne sait où. Ce qui était sûr, c'est qu'il n'allait pas plus vite qu'un bambin sur un tricycle. Cependant, Gertrude rentra chez elle - dans la grange - et pleura avant de décider d'un plan d'action.

Puis elle décida d'un plan d'action. Le plan se résumait à ceci :
\begin{itemize}
	\item Pleurer à nouveau une dizaine de minutes
	\item Sortir voir si Gérard était encore visible, ce qui semble probable au vu de sa vitesse et de sa corpulence
	\item Prendre le tracteur en panne de Roger le mécanichien
	\item Rouler vers la direction opposée à celle prise par les Bernard et Wilfried-Constantin, afin de se souvenir par où il est parti
\end{itemize}

Après avoir pris le soin d’inonder la grange, elle sortit donc. Sauf qu'elle y est restée 10 minutes et 1 seconde, et ce fut malheureusement trop tard pour apercevoir Gérard.

Elle pleura de nouveau, même si ça ne faisait pas partie de son plan génial.

Une voix lui parvint aux oreilles. C'était Gérard qui disait : \cg Ben... \cd. Ce n'était en fait que l'écho de ce que Gérard a prononcé il y a maintenant un quart d'heure, mais comme elle ne le sut jamais, elle le prit pour un signe de vie venant de son cher frère. Déterminée à le retrouver mort ou vif, elle chaussa ses sandales scandinaves et partit à la recherche du tracteur.

Il est important de préciser à ceux qui n'ont pas regardé la carte que Péguland est une toute petite île du Triangle des Bermudes ignorant totalement l'existence de nos continents. Mais par le plus grand des hasards, certains de nos produits de bonne facture sont perdus lors d'échanges commerciaux hautement sécurisés et sont retrouvés à Péguland. C'est pourquoi un jour, un habitant qu'il est inutile de nommer a jeté une bouteille à la mer contenant la carte présente en début du livre. La curiosité arriva sur les côtes japonaises avant d'être interceptée par un sous-marin russe. Elle devint mondialement connue. Cependant la carte a dû être passée aux rayons gamma car de l'eau s'est infiltrée dans la bouteille pour la simple et bonne raison que l'habitant n'avait pas pensé à mettre de bouchon. Malheureusement cette carte ne fournissait aucune indication utile pour trouver cette île, et jamais nos civilisations ne rencontrèrent la leur.

De retour à la plus célèbre - la seule en fait - ferme de Péguland, on peut voir Gertrude qui fait le tour de la propriété une vingtaine de fois. Aucun tracteur en vue - le comble pour une ferme. Elle décida de retourner pleurer dans la grange. En poussant la grande porte rouge, elle vit enfin le fameux tracteur en panne. Par chance, l'inondation de larmes provoqua un court-circuit qui fit démarrer le tracteur au moment où elle s'assit dessus. Roger qui passait dans le coin fut époustouflé de voir son tracteur redémarrer après 5 années d'immobilité. Aussi décida-t-il de remettre un peu d'essence, qu'il avait siphonnée afin d'éviter les accidents dans la ferme, pour éviter les accidents au milieu de la route. C'est en effet ce qui était arrivé à Roger : il est tombé en panne d'essence alors qu'il passait devant cette modeste ferme - qui couvre 70\% du pays. Ensuite il est tombé en panne tout court, et ne fut jamais capable de faire redémarrer le tracteur. Il prit donc son jerrican en plastique qui avait rouillé et s'approcha du tracteur. Sauf que le tracteur avait aussi disparu. Cette foi l'oie malfaisante s'appelait Gertrude.

Vaillamment, Roger enfourcha un cochon, attacha le jerrican avec un de ses cheveux à la queue du cochon, et partit poursuivre le tracteur en lançant : \cg Allez le cochon lô ! Faudlait rattlapper le tlecteur !\cd, ce sur quoi le cochon couina et fonça en direction du tracteur.

Comme tout le monde le sait, un cochon, aussi grassouillet soit-il, va bien plus vite qu'un tracteur, sauf si ce dernier a des jantes en laiton. Roger étant expert en rodéo de cochon, et le tracteur n'ayant carrément pas (et non \cg plus\cd) de jantes, il rattrapa le tracteur en moins d'un an. Pour préciser les choses, on pourra même dire moins d'un mois. Et pour les plus curieux à l'aise avec les nombres très petits, c'est très exactement en 1/32 768\textsuperscript{ème} de seconde qu'il rattrapa le tracteur et Gertrude. C'est du moins ce qu'indiquait sa montre à quartz de marque \cg Made in China\cd, qu'il a trouvée en train de suffoquer sur une plage de Péguland. C'était une rencontre émouvante et la petite chose avait avalé un sac en plastique, plutôt que de se ranger soigneusement dedans. Depuis ce jour, Roger garda précieusement la montre à sa cheville droite, même s'il s'est avéré que son quartz avait rendu l'âme.

\chapter{À la recherche de Gérard}
\section{Mais où est Gérard ?}
Gérard l'agriculteur est un flemmard de première (classe à laquelle il a arrêté ses études) : voici comment se découpe son emploi du temps\footnote{Oui j'aime les listes inutiles} :
\begin{itemize}
	\item 40\% de son temps à dormir dans son lit, dont 10\% avec Gertrude avant qu'elle aille sur le canapé, ne supportant plus ses ronflements
	\item 3,1415926535\% à s'occuper de la ferme
	\item 30\% en siestes en tout genres
	\item 50\% à manger et/ou à boire
\end{itemize}

Quelle ne fut pas sa joie lorsqu'il se réveilla - après une longue sieste durant le trajet - et découvrit que ses ravisseurs l'avaient abandonné dans un endroit tout bonnement charmant avec toute la nourriture qu'il aimait en abondance. Les Kidnapoules\footnote{Le narrateur se trompe tout du long, ce sont bien des poules, ce qu'ont justement remarqué les 2G\footnote{Les 2G est le surnom du couple formé par Gérard et Gertrude}}\footnote{Même si Gérard reste un grand gamin, il faudrait plutôt parler d'adultnapping\footnote{Ce mot n'existe pas officiellement\footnote{L'auteur s'est renseigné sur le sujet avant d'un informer le narrateur.}} ou de Gérarnapping. Donc plutôt que Kidnapoules, on pourrait dire Gérarnapoules, mais ça sonne très mal.}\footnote{Qui rappelons-le sont des oies sauvages.} ont eu tout leur temps libre en captivité pour observer Gérard en train de manger dans le champ derrière la ferme. Ainsi il avait droit à tout un étalage de chips, saucissons et bières de tous les coins de l'île, avec la mention \cg Qualité Carrouf\cd.

\section{Mais que fait Gérard ?}
Gérard est dans une grotte redécorée pour l'occasion du Sud-Ouest de l'île. À Troce pour être précis. En effet, tout ce gavage intensif avait pour but inavoué de lui faire exploser la panse et de le laisser agoniser dans un monticule de sachets de chips au saucisson parfumées à la bière.

\section[Le début de la quête]{Le début de la quête (Miaouss oui la quéqu---)}
\cg GERTRUDE ! beuglait le cochon sur lequel était debout Roger\\
-- Quoi Polky ?\\
-- GERTRUDE ! beuglait le type qui était debout sur le cochon\\
-- Quoi Loger ?\\
-- Où vas-tu ? demanda Roger avec son bel accent citadin\\
-- Dans l'autle dilection d'où qu'ils ont emmené mon mali pour me souv'nir d'où c'est. expliqua Gertrude\\
-- C'était pas plus simple de les suivre ? remarqua fort justement Roger\\
-- Oh ben j'y avions pô pensé ! dit Gertrude après avoir marqué une pause\cd

Et Gertrude arrêta le tracteur.

Et le cochon continua de foncer dans le tracteur.

Et Roger finit dans les bras de Gertrude, laquelle le laissa tomber dans la fange du pré fraîchement arrosé par des cadavres de bouteilles de bière. Les mêmes que celles que boit Gérard. Cela rendit Gertrude gravement triste. C'est pourquoi elle laissa tomber Roger dans la fange du pré fraîchement arrosé par des cadavres de bouteilles de bière, avant de se retourner dans le silence.

Le lendemain, Roger revint avec Porky voir Gertrude. Elle avait passé toute la nuit à pleurer et avait les yeux injectés de sang. Trois jours après, Roger en comprit la cause et nettoya le champ de toutes les bouteilles. Aussitôt Gertrude repris son sourire habituel - un joli poker-face. Mais quand il lui demanda ce qu'ils allaient faire pour Gérard, elle pleura de nouveau quatre jours, pendant que Roger préparait le matériel de survie afin d'être paré à toute éventualité. Il leur manquait malgré tout un objet indispensable : des serviettes. Roger a su qu'il leur en fallait non pas en lisant le \emph{Guide du voyageur galactique}, ouvrage dont seuls certains britanniques ont entendu parler, mais par pure intuition. Le problème est qu'il a oublié la sienne chez lui à Sion et qu'à la ferme on n'avait même pas l'eau chaude, donc l'idée de se laver n'était jamais venue jusqu'ici (quand Roger l'a compris on a eu droit à 3 tentatives de suicide).

Roger alla donc en acheter dans la ville la plus proche. Il rentra bredouille. Mais puisque nous ne sommes pas matérialistes, on évoquera son enrichissement spirituel : tout d'abord il revient avec l'idée d'essayer de songer à penser à prendre de l'argent, pour payer les serviettes. Ensuite, il a aussi l'infime joie d'avoir salué Gérard, qu'il n'a pas vu depuis plus d'une semaine. Joie dont il fit part à Gertrude à son retour.

Lorsqu'il revint de son second aller-retour, il avait enfin les serviettes. Accessoirement, il a de nouveau vu Gérard.

\cg Très bien ! Nous sommes prêt à aller chercher Gérard ? demanda gaiement Roger\\
-- Hum j'pensions qu'oui ! annonca Gertrude\\
-- D'acc on s'casse alors. répliqua Roger qui commence à intégrer la vulgarité de Gertrude. Allons-y let's go ! C'est parti les amis !\cd

Les ravages de la télévision et des dessins animés tout mignons tout roses...
\vfill
\emph{C'est sur ce début de recherche de Gérard que se clot se chapitre sur la recherche de Gérard, merci de votre lecture et à mardi prochain !}

\chapter{La chevauchée vers l'Ouest}
Lorsque Roger voulu chevaucher depuis la première fois en 5 ans son tracteur adoré, il se fit assommer par un mouton fermement maintenu par Gertrude, qui le laissa ensuite tomber dans la fange. Roger s'écroula non pas dans la fange, mais dans la boue et Gertrude en profita pour escalader le tracteur, ce magnifique tracteur rouge traité contre les puces - un problème souvent ignoré par les fermiers et qui cause des ravages chaque année.

À son réveil, la seule chose environnant Roger était son fidèle destrier Porky IV - le père du Porky de l'autre jour. Roger l'enfourcha et partit dans la direction opposée à celle prise auparavant. Cette fois-ci, il a rattrapé son tracteur au bout d'une bonne heure. En effet, bien que ce cochon-là soit très rapide, il a pris un retard immense d'au moins -2,5 fermes\footnote{La ferme est une unité de surface représentant la superficie de la ferme de Gérard, c'est à dire 70\% des 2 000 hectares de l'île, soit approximativement 14 km\textsuperscript{2} Les distances sont converties en fermes en multipliant par la \emph{constante de le champ de Gérard} notée $ConstanteDeLeChampDeG\acute{e}rard$ et qui vaut -1/12 m. Le résultat obtenu est ainsi en dimension supérieure, puisqu'on l'a alors multiplié par bien plus que l'infini. À l'inverse, il faut multiplier par -12 pour obtenir des mètres à partir de fermes. Notons toutefois que ces conversions n'ont pour eux aucun intérêt puisque les mètres n'existent pas, à part quand ils doivent calculer la surface d'un carré.} à cause de son évanouissement, car étrangement, la boue de Péguland fait s'évanouir quiconque plonge sa tête dedans, alors que la fange est beaucoup plus neutre. C'est pourquoi en l'an 12 AVC\footnote{AVant Gérard, il est faux d'écrire AVG, la barre du G a souvent été prise pour une saleté, à juste titre} une grande campagne de nettoyage fut lancée à travers l'île pour la débarrasser de toute sa boue, puis de répandre de la fange à la place.



\end{document}